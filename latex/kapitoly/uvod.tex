\chapter*{Úvod}
\addcontentsline{toc}{chapter}{Úvod}
\vspace{0.5cm}

{\noindent Tému „Oživil som počítač z pivnice" som si vybral preto, lebo už vyše dvoch rokov využívam staré počítače ako servery na spustenie zopár programov, ktoré mi uľahčujú život. Posledného pol roka už mojej mame spravujem server na ktorom hostujem webstránku pre jej firmu. Všetky servery ktoré využívam, by väčšina ľudí nazvala 'starými'. Väčšina z týchto počítačov bola odsúdená na zošrotovanie, no moja bývalá škola mi ich ponúkla za veľmi atraktívnu cenu, nula eur a nula centov. Takže si myslím, že táto téma je mi celkom blízka.}

Cieľom tejto práce je priblížiť vám proces využitia, prípadne aj vylepšenia starého počítača na server, na ktorom budete môcť spúšťať rôznorodé programy, napríklad web servery, ak sa chcete naučiť vytvárať webstránky, softvér na menežment hesiel, ak ich už máte naozaj veľa a samozrejme aj server na zdieľanie súborov na vašej domácej sieti. Dozviete sa nielen ako server a softvér na ňom nakonfigurovať, ale aj základné príkazy pre operačný systém Linux a hlavne ako server zabezpečiť, aby sme predišli hackerským útokom. Samozrejme, že vám nemôžem ukázať, ako nainštalovať a nastaviť všetok softvér na svete, takže vám prezradím aj ako vyhľadať pomoc, aby ste si ďalší server zvládli nakonfigurovať sami. Toto všetko sa pokúsim vysvetliť jednoducho a zrozumiteľne aby čítateľ nemal pocit, že to je nemožné. Lebo v svete informačných technológii je len málo vecí, ktoré sú nemožné. Takisto vysvetlím, prečo som vôbec zvolil operačný systém Linux a nie Microsoft Windows.

Pokiaľ vás tento úvod zaujal, čítajte ďalej. Myslím si, že nebudete sklamaní. V každom prípade musím ešte poďakovať môjmu konzultantovi, pánovi profesorovi Tiborovi Pálszegimu za pomoc pri písaní tejto práce, mojim rodičom za motiváciu, a v prvom rade vám, čítateľovi, ktorý sa rozhodol prečítať si túto prácu.

\clearpage
