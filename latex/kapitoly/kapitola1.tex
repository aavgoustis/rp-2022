% \chapter*{Viem použiť môj počítač ako domáci server?}
% \addcontentsline{toc}{chapter}{Viem použiť môj počítač ako domáci server?}
% \vspace{0.5cm}

Keď vám už nestačí výkon vášho starého počítača, alebo ste konečne podľahli vábeniu novej technológie, čo bude so starým počítačom?

Ak ste na tom podobne ako ja, možno si nájde trvalé bydlisko na povale, alebo sa schová niekde do kúta mimo dosahu\ldots aspoň dovtedy, kým ho nevyužijem na náhradné diely alebo sa ho natrvalo nezbavím. Ale nemusí to tak byť.

Váš starý počítač je Linuxový server, o ktorom ste nikdy nevedeli, že ho máte. A váš starý hardvér, ktorý už nepostačuje na každodenné použitie, je jednoduché premeniť na plne funkčný server, ktorý budete môcť využiť na učenie o počítačoch a Linuxe, alebo aj na self-hosting.

\section*{Minimálne systémové požiadavky}
\addcontentsline{toc}{chapter}{Minimálne systémové požiadavky}

Predtým ako spustíte svoj Linux server, musíte najprv skontrolovať špecifikáciu. Linux je vybavený tak, aby fungoval aj na tých najpomalších starých zariadeniach, takže to zvyčajne nie je problém ak premienate počítač z obdobia približne posledných 10 rokov, ale pred začatím je nevyhnutné skontrolovať, či máte základné vybavenie.

Jednou z vecí, ktorá prekvapí nových používateľov Linuxu, ktorí prišli z iných operačných systémov, ako sú Windows alebo Mac OS je, že jeden celok s názvom Linux os je vlastne distribúcia Linuxu. Koncept distibúcie Linuxu (Linux distribution) je veľmi populárny a bežne sa označuje ako Linux distro. \autocite[3]{castro_linux-distros_2016}

Linux nie je len operačný systém pre domáci počítač; existujú aj iné prostredia, v ktorých sa používa. Okrem toho existuje mnoho distribúcií, ktoré boli vytvorené len s ohľadom na konkrétnu úlohu.



% ===========================================================
\section*{Viem použiť môj počítač ako domáci server?}
\addcontentsline{toc}{chapter}{Viem použiť môj počítač ako domáci server?}

Áno. Môžete použiť svoj starý počítač ako server na ukladanie a zdielanie súborov, ako médiové centrum, alebo na hostovanie web stránky. Takisto ho môžete použiť ako herný server.

% \chapter*{Aký je rozdiel medzi PC a serverom?}
% \addcontentsline{toc}{chapter}{Aký je rozdiel medzi PC a serverom?}
\section*{Aký je rozdiel medzi PC a serverom?}

PC (Personal Computer) je osobný počítač využívaný jednotlivcom pre osobné využitie. Narozdiel od PC, server je počítač určený na spúštanie sieťových služieb. To znamená, že je zodpovedný za vybavovanie žiadostí od iných počítačov v sieti a poskytuje im zdroje, ktoré potrebujú. Servery sa často používajú vo firmách a organizáciách na poskytovanie zdieľaného prístupu k súborom, aplikáciám alebo iným zdrojom.

\section*{Aké sú výhody premeny starého počítača na server?}

Premena starého počítača na server má niekoľko výhod. Môžete si tak vytvoriť domácu sieť a mať centralizované úložisko pre všetky vaše súbory. Server môžete tiež použiť na zdieľanie internetového pripojenia s ostatnými zariadeniami u vás doma, alebo si dokonca založiť malú kancelársku sieť.

Servery sú jedným z najdôležitejších aspektov počítačového sveta. Ukladajú a spravujú údaje, poskytujú bezpečnosť a autentifikáciu a zabezpečujú plynulý vchod všetkého. Ak chcete počítač premeniť na server, potrebujete počítač s určitými špecifikáciami. Potom môžete použiť rôzne programy na premenu počítača na server.

\section{Čo potrebujem na premenu počítača na server?}

Na premenu počítača na server budete musieť nainštalovať softvér a upraviť nejaké nastavenia.

Keď chcete použiť svôj počítač ako server, potrebujete otvoriť 
